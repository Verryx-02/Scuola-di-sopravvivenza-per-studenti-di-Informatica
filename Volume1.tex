\documentclass[18pt]{extarticle}
\usepackage[a4paper, top=2cm, bottom=2cm, left=2cm, right=2cm]{geometry}

\usepackage[
    colorlinks=true,
    linkcolor=black,        % Link interni (indice, riferimenti) in nero
    urlcolor=blue,          % Link esterni (URL) in blu
    citecolor=black,        % Citazioni in nero
    filecolor=black         % File in nero
]{hyperref} % For the Table of Contents

\usepackage{enumitem} % For lists

\usepackage{graphicx}
\usepackage{float}

\usepackage{xcolor}

\title{Scuola di sopravvivenza per Informatica volume 1}

\author{Verryx-02}
\date{2025/08/15}
\begin{document}

\maketitle

% Table of Contents (acts as your index)
\tableofcontents
\newpage

\section{Introduzione}

\subsection{Contenuto della guida}
In questa guida troverai i consigli più disparati per sopravvivere ad Informatica. Ti consiglio di leggerla tutta, potresti trovare un po' di cose interessanti.\\
Questo è il primo documento che scrivo il \LaTeX, perciò abbiate pietà di me.

\subsection{Chi sono io}
Io, sono \href{https://github.com/Verryx-02}{\textbf{Verryx-02}}, uno studente di Informatica fuori sede. 
Per l'esattezza, mentre sto scrivendo questo documento sono all'ultimo anno della triennale qui all'Università di Udine.\\
Da ottobre 2025 sono rappresentante degli studenti del Dipartimento di Scienze Matematiche, Informatiche e Fisiche, ma soprattutto,\\
\textbf{sono uno che ha fatto tutti gli errori possibili.}


\subsection{A chi sono rivolti questi consigli}
\textbf{A tutti.}\\
Ma in particolare a chi si sente in difficoltà, ha dubbi, o fatica a trovare il proprio metodo.\\
Se pensi che l'università sia un ostacolo insormontabile, forse la stai affrontando nel modo sbagliato e questi consigli potrebbero farti comodo.\\
Non ti mentirò. Informatica qui a Udine è dura, ma ce la puoi fare se te la giochi bene.\\
Se sei di IBML, alcuni di questi consigli sono adatti anche a te visto che hai degli esami in comune con Informatica. 


\subsection{Perché ho scritto questo documento}
Per evitare che altri commettano gli stessi errori che ho fatto io.  
Credo nella filosofia dell’open source: condividere ciò che imparo per me è un dovere se può rendere la vita più facile a chi verrà dopo di me.

\subsection{Perché dovresti fidarti di me}
Perché io sono partito da zero.  
Vengo dall' Istituto Tecnico Agrario di Avellino, a 800 chilometri lontano da Udine. 
A scuola non ero brillante, non avevo voglia di studiare e la mia insegnante di matematica non era un gran ché.  
Il mio livello in matematica al primo anno di Informatica si può riassumere così:
\begin{itemize}
\item non sapevo disegnare $y = 2x$,
\item non sapevo risolvere le equazioni,
\item non sapevo cosa fosse una funzione,
\item non avevo mai sentito parlare dei logaritmi,
\item non ero in grado di superare il TOLC-S / TOLC-I, 
\item il mio primo anno di università nonostante tutto l'impegno, ho superato 0 esami. 
\end{itemize}%
Ho dovuto adattarmi, capire quali fossero i problemi e trovare soluzioni fuori dal comune. 
Perciò puoi credermi: \textbf{Se ce l'ho fatta io, puoi farcela anche tu.}

\clearpage
\section{Disclaimer}
Questa guida è frutto dell'esperienza mia e di un ristretto gruppo di studenti di Informatica all'Università di Udine. Tutti con obiettivi, conoscenze pregresse, interessi e titoli di studio differenti.
Ti invito a non prendere il tutto come oro colato, ma allo stesso tempo, a non sottovalutare questi consigli.


\subsection{Ambito di applicazione}
Tutte le informazioni in questa guida valgono specificamente per:
\begin{itemize}
    \item \textbf{Corso di laurea in Informatica} (Università di Udine)
    \item \textbf{IBML} (per gli esami in comune)
\end{itemize}
Quando parlo di ``università'', mi riferisco esclusivamente a questi due corsi. Non posso parlare per altri corsi di laurea o altri atenei perché non li ho frequentati.


\subsection{Responsabilità}
Ho scritto questa guida di mia sponte, per condividere ciò che ritengo di aver imparato.\\
Mi assumo le piene responsabilità, delle conseguenza che questa guida può avere.\\
Non ha nessun legame con istituzioni accademiche.\\
Non rappresenta in alcun modo NEXT (partito studentesco di cui faccio parte).\\
I miei peer reviewer hanno contribuito solo con feedback e correzioni, sono pertanto esenti da ogni responsabilità.\\

\section{Contributi e aggiornamenti}
Le informazioni sono aggiornate ad \textbf{agosto 2025}. Alcuni dettagli potrebbero cambiare nel tempo.\\
Per tutti (studenti e docenti): Se questa guida ti infastidisce per qualche motivo, se noti informazioni obsolete, o semplicemente vuoi contribuire con miglioramenti, sentiti libero di:
\begin{itemize}
    \item Aprire una \href{https://github.com/Verryx-02/Scuola-di-sopravvivenza-per-studenti-di-Informatica/issues}{issue}
    \item Fare una \href{https://github.com/Verryx-02/Scuola-di-sopravvivenza-per-studenti-di-Informatica/pulls}{pull request}
\end{itemize}
Proposte di modifica e critiche sono ben accette se giustificate. Assicurati solo di seguire le indicazioni del file \href{https://github.com/Verryx-02/Scuola-di-sopravvivenza-per-studenti-di-Informatica/blob/main/CONTRIBUTING.md}{CONTRIBUTING.md}
\begin{quote}
Si assume che sappiate scrivere in \LaTeX; altrimenti, \href{https://www.learnlatex.org/it/}{sappiatelo}. \textasciitilde M.M.
\end{quote}

\clearpage
\section{Quale PC comprare per studiare all’università}
Per studiare Informatica, ti servirà un portatile fin dal primo anno.\\
Che tu lo voglia o no, dovrai scrivere codice ed eseguirlo.\\
In alcuni corsi il PC è richiesto esplicitamente:
\begin{itemize}
\item architettura degli Elaboratori,
\item programmazione e Laboratorio,
\item sistemi Operativi,
\item laboratorio di Algoritmi e Strutture Dati,
\item laboratorio di Basi di Dati,
\item linguaggi di Programmazione
\end{itemize}
Ma anche per tutti gli altri corsi, avere un portatile è \textbf{molto comodo}: per leggere slide, prendere appunti, e organizzare i materiali.


\subsection{La mia opinione sui portatili di oggi}
Da appassionato di open source e configurazioni PC mi brucia un sacco dirlo, ma...
Al momento, \textbf{non esiste} un portatile Windows che batta un \textbf{Macbook Air M1 base} per uso universitario.\\
Il Macbook Air M1 del 2020 (8GB RAM, 128GB SSD) è ancora oggi, nel 2025, super competitivo.\\
E ha alcuni vantaggi \textbf{oggettivamente imbattibili}:
\begin{itemize}
\item è leggero e comdodo, puoi usarlo ovunque,
\item è compatibile con tutto ciò che serve nei corsi,
\item rimane stabile e fluido anche dopo anni di utilizzo,
\item il terminale integrato basta e avanza per tutto il triennio,
\item puoi comunque installare VS Code, Git, Python, Java, Scheme, ecc.
\item ha una batteria senza senso che dura anche 2 giorni con uso leggero,
\item puoi rivenderlo facilmente perché tiene bene il suo valore nel tempo,
\item non scalda, il che porta ad una minore usura dei componenti nel tempo.
\end{itemize}


\subsection{Fascia di prezzo consigliata}
In generale, gli M4 con 16GB di RAM su Amazon arrivano spesso intorno ai 900 euro.  
È tanto, è vero, ma ti durerà anni e lo potrai rivendere facilmente se lo tratti decentemente.\\
Considera anche il prezzo di una cover e di una borsa per il pc. 
Per quanto riguarda gli M1, si trovano a molto meno e come ti ho detto, sono ancora perfettamente validi. 
\begin{quote}
\textbf{Non disdegnare i Mac usati}, c'è un motivo se hanno tanto mercato. Può valerne la pena. Informati e fai le tue considerazioni.
\end{quote}


\subsection{Cosa non comprare assolutamente}
\textbf{Se scegli un MacBook:}
\begin{itemize}
\item Evita qualunque Macbook fuori dal modello Air base.
\item Non ti serve più potenza di quella offerta da un M1.
\item Non aggiungere memoria in fase di configurazione, usa un ssd esterno.
\end{itemize}%
\textbf{Evita i pc con processori ARM:}
\begin{itemize}
\item non sono ancora maturi (inoltre non conosco nessuno che ne abbia uno quindi non mi sento di consigliarli)
\item Per quanto mi riguarda, se non ha Intel o AMD o non è Apple lo sconsiglio.
\end{itemize}%
\textbf{Evita i portatili da gaming:}
\begin{itemize}
\item sono eccessivamente pesanti ed ingombranti,
\item rumorosi,
\item con una autonomia reale ridicola (circa 2-3 ore)
\end{itemize}%
Evita questi tipi di PC se il tuo obiettivo principale è studiare Informatica. Possono avere senso \textbf{solo} se prevedi di usarli anche per gaming o rendering.


\subsection{Ma se proprio vuoi un portatile Windows o Linux}
Cerca un portatile con:
\begin{itemize}
\item \textbf{processore AMD Ryzen 5 o Intel i5} o superiore, generazione recente ma anche quelle più vecchie vanno bene.
\item \textbf{16GB di RAM}
\item \textbf{SSD da almeno 256GB}
\item \textbf{buona autonomia} (7+ ore \textbf{reali} e testate da recensori)
\item \textbf{ricarica tramite USB-C} così puoi caricarlo con un powerbank.
\end{itemize}%
Se scegli Windows, \textbf{preparati a installare Ubuntu} in dual boot, su macchina virtuale o ad usare la WSL: ti servirà per Sistemi Operativi e molti esercizi in bash o C.\\
Ti consiglio di dare un'occhiata ai \textbf{Lenovo ThinkPad}. In generale sono delle ottime macchine. Costano un pò ma hanno i loro vantaggi:
\begin{itemize}
\item community enorme
\item driver per linux ottimizzati scritti da Lenovo
\end{itemize}%
Comunque in generale è un mercato enorme, qualcosa si trova sempre, anche se hai poco budget.


\subsection{Nota per chi vuole entrare nei MadrHacks}
I Mac sono ottimi per l’università, ma potrebbero darti qualche problema se vuoi entrare nei \textbf{MadrHacks} (il gruppo di cybersecurity dell’ateneo).\\
In alcuni casi, il sistema chiuso di MacOS può renderti la vita più difficile.\\
Se sei sicuro che vorrai far parte dei MadrHacks considera un portatile Linux-native.\\
Ti consiglio di guardare i Pc Tuxedo o Lenovo Thinkpad o qualsiasi altra cosa che rispetti i requisiti indicati qui sopra.


\subsection{Esperienza personale}
Io uso un \textbf{Macbook Air M1 base (8GB RAM, 128GB SSD)} da metà del primo anno.\\
Funziona \textbf{perfettamente} per tutti i corsi.\\
Unico problema avuto: \textbf{ArmSim}, che ora non si usa più (sostituito da un simulatore online).\\
Ancora oggi le performance sono ottime e la batteria mi dura tutto il giorno (a volte anche di più)\\
Infine, ti invito a trarre le tue conclusioni.\\
Tieni presente che:
\begin{itemize}
\item io stesso non sono un fan della Apple,
\item il logo sul mio pc è coperto dallo sticker di una pizza,
\item nessuno mi paga o mi obbliga a darti questi consigli.
\end{itemize}%
Se non ti fidi, cerca recensioni in giro e fatti la tua idea.\\
Questo è un mondo che cambia velocemente, perciò informati al meglio.\\
Se non sai come informarti ti consiglio alcuni dei miei canali italiani preferiti. Puoi partire da qui:\\
\href{https://www.youtube.com/@Prodigeek}{\textbf{Prodijeek}}, \href{https://www.youtube.com/@MoreThanTech}{\textbf{MoreThanTech}}, \href{https://www.youtube.com/@SaddyTech}{\textbf{SaddyTech}}, \href{https://www.youtube.com/@Blink46yt}{\textbf{Blink46}}.

\clearpage
\section{La differenza tra Università e Scuole Superiori}
Università e superiori sono \textbf{due mondi completamente diversi}.\\
Molti credono di poter affrontare l'università con lo stesso approccio usato alle superiori.\\
\textbf{Si sbagliano.}\\
L'università è un ambiente in cui nascono \textbf{legami profondi}, basati su \textbf{collaborazione e supporto reciproco}.\\
La vera differenza però è un'altra: la maggior parte delle persone \textbf{ha scelto} di essere qui. E se non ci vuole stare, può scegliere di andare via.\\
Questo crea un contesto unico: un'aula piena di studenti con \textbf{interessi simili ai tuoi}, motivati e spesso \textbf{appassionati} di ciò che studiano.\\
Ma la cosa più importante da capire è questa:
\begin{quote}
\textbf{All'università non si studia come alle superiori.}
\end{quote}%
Qui non basta seguire le lezioni e studiare un'oretta al giorno. Servono:
\begin{itemize}
\item \textbf{metodo},
\item \textbf{costanza},
\item \textbf{un gruppo di studio solido},
\item \textbf{massimo impegno}.
\end{itemize}


\section{A cosa serve un gruppo di studio}
Un buon gruppo di studio ti aiuta a:
\begin{itemize}
\item ridurre il carico cognitivo di quello che studi,
\item studiare in modo più stimolante e divertente,
\item risolvere problemi che ti sembrano impossibili,
\item scoprire metodi di studio alternativi,
\item trovare e condividere informazioni in modo più rapido,
\item creare legami forti, anche oltre l'università,
\item scoprire nuovi interessi inaspettati.
\end{itemize}
\begin{quote}
L'università senza un gruppo di studio è come una torta di mele bruciacchiata senza mele: \textbf{una merda.}
\end{quote}
Non avere paura di cambiare gruppo se quello attuale non fa per te. Ci sono un sacco di studenti con cui fare amicizia, anche degli anni successivi.\\
Se il tuo gruppo è troppo grande e studiare diventa difficile, parlane con gli altri liberamente.\\
Il gruppo può essere spezzato durante le sessioni di studio e riunito durante le pause per avere il meglio sempre.


\subsection{Come trovare un buon gruppo}
Non ho grandi consigli su questo argomento. Dipende molto da te.
Posso dirti che io ho fatto queste 3 cose principalmente che mi hanno aperto delle porte:
\begin{itemize}
\item condividere i miei appunti con chi ne aveva bisogno, 
\item andare a parlare con chi mi sembrava più interessante,
\item sedermi alle prime file, sperando di trovare ragazzi molto interessati alla lezione.
\end{itemize}

\clearpage
\section{Cosa fare appena iniziano i corsi}
\begin{itemize}
\item \textbf{Unisciti al server \href{https://discord.com/invite/pHHv6TE8WW}{Discord} di Informatica} e iscriviti ai corsi che ti interessano (meglio se ti unisci a tutti, tanto prima o poi ti serviranno)\\
  Troverai appunti, vecchie prove, consigli pratici.\\
  Se non sai come fare, chiedi ai ragazzi degli anni successivi.
\item \textbf{Iscriviti ai \href{https://elearning.uniud.it/moodle/login/index.php}{Moodle} dei corsi del tuo anno} e a quelli degli anni precedenti.\\
  Troverai una marea di info importanti, materiali, e in certi casi (vedi Logica Matematica) l'accesso all'esame.
\item \textbf{Scarica \href{https://play.google.com/store/apps/details?id=it.easystaff.uniud&hl=it}{EasyUniud}.}\\
  Ti mostra il calendario delle lezioni aggiornato e le aule occupate e libere in cui puoi andare a studiare.
\item \textbf{Entra nei gruppi Whatsapp di informatica}\\
Ce ne sono parecchi, lo so. È un problema da sistemare. 
\end{itemize}


\section{Cosa fare alla prima lezione di ogni corso}
\begin{itemize}
\item Siediti vicino alla lavagna. Ti consiglio la terza fila dal basso: è abbastanza vicina per vedere bene alla lavagna, ma abbastanza lontana da non farti venire il torcicollo se devi leggere le slide proiettate.
\item Annota le \textbf{modalità d'esame}, i \textbf{contatti del docente}, e ogni altra informazione utile.
\item Non parlare durante la lezione. A causa della forma dell'aula il professore sente praticamente tutto.
\item Se hai domande pertinenti, falle pure, ma \textbf{evita il brusio} di fondo.
\item Se non ti vengono date informazioni sugli esami, sentiti libero di chiederle.
\end{itemize}


\section{Cosa fare se non stai capendo a lezione}
\begin{itemize}
\item Dillo al prof: Pensa 30 secondi a come formulare la domanda, alza la mano e chiedi quello che vuoi con rispetto.  
  Se il docente non risponde subito, lo farà durante la pausa o a fine lezione.
\item Parlane con i tuoi compagni, scoprirai che non sei l'unico a non aver capito nulla.
\end{itemize}


\section{Quando ha senso andare a lezione}
C'è un'idea diffusa: \textit{``Per passare l'esame devi seguire tutte le lezioni.''}\\
\textbf{Non è sempre vero.}\\
Io, come molti altri, ho capito che esiste una regola d'oro:
\begin{quote}
\textbf{Se a lezione non capisci una sega, è meglio non andarci.}
\end{quote}
Nella maggior parte dei casi, i professori non aspettano te. Vanno avanti a spiegare cose che tu non potrai capire perché non hai capito la lezione precedente.\\
Andare a lezione in quel caso diventa una completa perdita di tempo.\\
Puoi usare quel tempo per studiare meglio da solo o con il tuo gruppo, con la dovuta calma e capire le cose con i tuoi tempi.\\
L'università mette a disposizione tutti i mezzi di cui puoi aver bisogno.\\
Se ci sono risorse alternative che puoi usare (e ce ne sono), \textbf{usale.}
\begin{quote}
\textbf{Ma attenzione:} vai \textbf{sempre} alle prime tre lezioni di ogni corso.
\end{quote}
\begin{itemize}
\item La prima lezione serve a capire come sarà il corso, chi è il professore, come funziona l'esame, come ottenere i materiali e un sacco di informazioni fondamentali.
\item Le altre due servono a capire se ti conviene seguire o meno.\\
   Se rimani indietro e non riesci a recuperare, forse c'è qualcosa che non va su cui devi intervenire prima possibile.
\end{itemize}


\section{Guardare le video-lezioni}
Se non puoi andare a lezioni di persona o hai bisogno dei tuoi tempi, ti consiglio di guardare le lezioni registrate. In generale, quelle del periodo del Covid sono ancora utili oggi, alla fine i corsi non cambiano più di tanto. Ti invito comunque a fare le tue considerazioni.\\
Se scegli di guardare le video-lezioni, puoi:
\begin{itemize}
\item mettere in pausa quando vuoi,
\item riascoltare ciò che non ti è chiaro mille volte, finché non lo capisci, 
\item seguire la lezione quando vuoi e non dover sottostare agli orari imposti dall'università, 
\item concentrarti a capire le cose con calma al posto che scrivere ogni parola del prof. 
\end{itemize}%
Però devi fare attenzione, perché le video-lezioni possono essere un'arma a doppio taglio:
\begin{itemize}
\item perdi il contatto con gli altri studenti, e con i professori: siete solo tu e il tuo schermo
\item ti distrai più facilmente perché nulla ti impedisce di abbandonare la lezione in qualunque momento.
\end{itemize}%
Le lezioni registrate sono un mezzo potente, ma vanno usate nel modo giusto, facendo attenzione a non incappare nei problemi sopra citati.


\section{Ricopiare è il modo peggiore di studiare}
Se sei al primo anno ti renderai presto conto che non hai molto tempo per studiare, quindi quel poco che hai, non devi sprecarlo a fare cose inutili.\\
\textbf{Ricopiare le slide}, ad esempio, è un lavoro lunghissimo e totalmente inutile.\\
Ti fa sentire soddisfatto di aver fatto uno sforzo enorme e ti dà la falsa sensazione di aver capito o imparato tutto. 
Beh \textbf{non è così}.\\
Io (e molti altri come me) il mio primo anno ho ricopiato tutte le slide di Architettura degli Elaboratori.\\
Ho riempito quaderni di robe su quel corso spendendoci una quantità di tempo allucinante.\\
Ho seguito il laboratorio cercando di fare quello che riuscivo.\\
Poi all'esame ho preso 4/30. Perciò \textbf{fidati}. È inutile.
\begin{quote}
\textbf{A meno che tu non abbia una memoria fuori dal comune, ricopiare le slide non serve a nulla}.
\end{quote}%
La cosa veramente utile è \textbf{capirne il significato} e soprattutto \textbf{fare esercizi} su quello che studi.\\
Se non sai come studiare, ti consiglio queste due tecniche che con me e il mio gruppo hanno funzionato molto bene.\\
La loro valenza è scientificamente provata e ben documentata quindi non mi dilungherò più di tanto.


\section{Tecniche per migliorare lo studio}
\subsection{La tecnica del pomodoro}
Forse ne avrai già sentito parlare, la tecnica del pomodoro consiste nello studiare per un certo tempo, fare una breve pausa per poi riprendere lo studio.\\
Il mio consiglio è di fare sessioni di studio da 40 minuti seguite da una pausa di 10 minuti.\\
Dopo 4 sessioni, una pausa lunga di 30 minuti.\\
Questo per me è il tempo ideale, ma sentiti libero di modificarlo seguendo questa regola:
\begin{quote}
\textbf{Se dopo gli n minuti di studio senti che la tua attenzione è troppo bassa e ti senti stanco, devi ridurre i minuti di studio.}
\end{quote}
Ricorda che se diminuisci il tempo di studio, devi diminuire anche quello della pausa in maniera proporzionale.\\
\textbf{Usa il timer} del telefono per tenere sotto controllo il tempo, e quando il timer suona, stacca. Molla tutto, vai a farti un giro per 10 minuti.\\
Esplora l'università, scopri le backrooms con i girasoli con le luci UV sottoterra e le cassette entomologiche ai piani superiori.
\begin{quote}
\textbf{Sii curioso, ma non metterci più di 10 minuti.}
\end{quote}
Forse non ci crederai, ma in quei 10 minuti di pausa il tuo cervello elabora le informazioni apprese nei 40 minuti precedenti.
\begin{quote}
\textbf{La tecnica del pomodoro funziona molto meglio se la usa tutto il gruppo. Rende lo studio più efficiente e la pausa più divertente.}
\end{quote}%


\subsubsection{Benefici della tecnica del pomodoro}
La tecnica del pomodoro:
\begin{itemize}
\item ti fa fare un po' di movimento,
\item aiuta ad assimilare e fissare i concetti,
\item riduce la tensione e lo stress in sessione,
\item ti sveglia dallo stato catatonico se stavi facendo uno studio distratto,
\item favorisce un degrado più lento dell'attenzione migliorando l'efficienza dello studio.
\end{itemize}
Beh in effetti la pausa funziona a patto che tu non tenga impegnato il cervello riempiendolo di stimoli come quando scrolli su Instagram. In quel caso ti sentirai solo più stanco.
\begin{quote}
\textbf{Se vuoi che la tecnica del pomodoro funzioni, durante la pausa (e lo studio) devi staccare da tutto.\\
Non devi usare quel cazzo di telefono.}
\end{quote}
Se vuoi saperne di più sulla tecnica del pomodoro, dai un'occhiata \href{https://it.wikipedia.org/wiki/Tecnica_del_pomodoro}{qui}


\subsection{Le flashcards}
Che tu abbia una buona memoria o no, se vuoi memorizzare enormi quantità di informazioni, la tecnica delle flashcards è probabilmente la più efficace.\\
Prendi un cartoncino bianco, scrivi da un lato una domanda e dall'altro lato la risposta.\\
Ecco, quel cartoncino è una flashcard. Non ti resta che scrivere tutte le domande che ti vengono in mente su quello che stai studiando.\\
Quando avrai finito di scrivere le domande e le risposte potrai iniziare a ripeterle:\\
Leggi la domanda e prova a rispondere, poi leggi la risposta e cerca di capire cosa hai sbagliato.\\
Se non hai saputo rispondere o hai risposto male, metti la flashcard tra le prossime da leggere.\\
Se hai saputo rispondere correttamente, metti la flashcard in fondo al mazzo.\\


\subsubsection{Perché le flashcards funzionano così bene}
Perché ti fanno fare uno studio molto più attivo rispetto alla sola lettura/ricopiatura passiva delle informazioni.  
Ti obbliga a pensare e a fissare le informazioni.
\begin{quote}
\textbf{Le flashcards ti sbattono in faccia il fatto che nonostante tu creda di aver capito i concetti e li abbia scritti su un cartoncino, comunque non li hai memorizzati e di fatto non li sai}. 
\end{quote}
In più fanno molto comodo arrivati alla fine delle lezioni perché ti permettono di ripassare tutto molto velocemente.  
Può essere una attività divertente e stimolante crearle e studiarle in gruppo.  
Può creare dei momenti unici che a loro volta aiutano a fissare le informazioni. 


\subsubsection{Se non vuoi comprare i cartoncini}
Devi sapere che esiste una applicazione per cellulare e per pc molto comoda per scrivere e ripassare le flashcards, chiamata Anki.\\
È dotata di un algoritmo facilmente customizzabile studiato appositamente per massimizzare la memorizzazione.\\
Ti consiglio vivamente di darci un'occhiata.\\
Il grande vantaggio delle flashcards scritte con Anki è che sono condivisibili con chiunque decida di installare Anki.\\
Se come me vuoi aiutare quelli che verranno dopo di te, ti invito a condividere il tuo pacco di flashcards creato con tanta cura.\\
Io per primo ho condiviso le mie flashcards di Sistemi Operativi sul canale Discord.\\
All'epoca non conoscevo ancora Anki quindi sono semplicemente in Markdown.\\


\subsection{Il pisolino post-pranzo}
Se dopo pranzo fai fatica a carburare e in generale non arrivi a fine giornata con energia sufficiente, forse dovresti provare il \textbf{power nap}.\\
Un pisolino di 15 minuti può fare la differenza tra una giornata produttiva e una passata a fissare i libri senza combinare nulla.\\
Se non hai mai provato il power nap, provalo per una settimana. Potresti metterci un pò a trovare i tuoi tempi ideali.
\begin{quote}
\textbf{Non sottovalutare il power nap.} Molti studenti che conosco hanno migliorato significativamente la loro produttività pomeridiana solo grazie a un pò di riposo strategico.
\end{quote}


\subsubsection{Come funziona il power nap}
\textbf{Durata ottimale: 10-20 minuti massimo.}\\
Devi riuscire a non entrare nella fase del sonno profondo (slow-wave sleep), altrimenti ti sveglierai più stanco di prima per un fenomeno chiamato ``sleep inertia''.\\
\textbf{Tempistiche che funzionano secondo gli studi scientifici:}
\begin{itemize}
\item \textbf{10 minuti:} Il più efficace secondo uno studio della Flinders University - migliora immediatamente attenzione e performance cognitive
\item \textbf{15-20 minuti:} Range ottimale per evitare la sleep inertia
\item \textbf{Oltre i 20 minuti:} Alto rischio di entrare nel sonno profondo e quindi di svegliarsi peggio di prima
\end{itemize}%
\textbf{Metti la sveglia sempre}, anche se pensi di non addormentarti. Mettene più di una per essere sicuro.


\subsubsection{Benefici del power nap}
Un power nap fatto bene:
\begin{itemize}
\item Migliora la concentrazione per le successive 2-4 ore
\item Riduce lo stress e l'irritabilità
\item Aumenta la memoria a breve termine e la consolidazione dei ricordi (già dopo 6 minuti secondo uno studio dell'Università di Düsseldorf)
\item Migliora performance cognitive e tempi di reazione
\item Ti fa arrivare a sera senza essere uno zombie
\item Secondo alcuni studi, chi fa power nap regolarmente ha il \textbf{37\% in meno di rischio} di malattie cardiache
\end{itemize}
\textbf{Attenzione: Non sostituisce il sonno notturno. Se dormi male la notte, il problema è lì.}
\begin{quote}
\textbf{Fonte:} \href{https://en.wikipedia.org/wiki/Power_nap}{Wikipedia - Power Nap} per tutti i riferimenti scientifici.
\end{quote}


\subsubsection{Alternativa: il coffee nap}
Personalmente preferisco non dipendere dalla caffeina, ma se a te non importa puoi provare così:
\begin{itemize}
\item bevi un caffè velocemente
\item mettiti \textbf{subito} a dormire per 15 minuti massimo
\item la caffeina fa effetto proprio quando ti svegli
\end{itemize}
Sembra una cazzata, ma funziona perché la caffeina ci mette un po' per entrare in circolo.\\
È risultata \textbf{la tecnica più efficace} per ridurre gli incidenti di guida in soggetti con privazione del sonno.


\subsubsection{Quando non funziona}
Il power/coffee nap non è per tutti:
\begin{itemize}
\item se hai problemi di insonnia, evita di dormire di giorno
\item se il pisolino ti rende ancora più assonnato, prova a ridurre la durata a 5-10 minuti
\end{itemize}


\section{Prendere appunti a lezione}
\subsection{Prendere appunti a lezione}
Copiare tutto ciò che il docente scrive alla lavagna non solo è faticoso, ma spesso è \textbf{una perdita enorme di tempo e concentrazione}.\\
È l'equivalente di ricopiare le slide, ma peggio, perché qui hai poco tempo prima che le informazioni spariscano dalla lavagna.\\
Questo approccio ti distrae, frammenta la tua attenzione e ti impedisce di \textbf{seguire davvero} quello che il prof sta spiegando.\\
Inoltre ti toccherà riscrivere gli appunti in bella dopo la lezione se vuoi avere un lavoro decente, e dovrai farlo per tutte le lezioni che segui.
\begin{quote}
\textbf{Prendere appunti ha i suoi vantaggi se lo fai nel modo corretto, rielaborando i contenuti senza ricopiarli identici a come li scrive il docente.}
\end{quote}


\subsection{Alternative al prendere appunti a lezione}
Se non riesci a seguire e scrivere insieme, o se ti accorgi che prendere appunti ti rallenta troppo, prova una di queste alternative:
\begin{itemize}
\item Usa gli appunti già fatti da altri: Ce ne sono in giro su Discord. Se non li trovi puoi chiedere nei gruppi Whatsapp. In generale chiedi a chi ha già fatto quell'esame.
\item Stampa le slide e prendi appunti direttamente lì: Questa è forse la cosa migliore che puoi fare se il professore spiega usando le slide.\\
  Così non perdi tempo a riscrivere cose già scritte dal prof, ma ti concentri su spiegazioni, esempi e commenti extra.\\
  \textbf{Hint:} Se ti serve più spazio sulle slide per scrivere, rimpiccioliscile prima di stamparle, così avrai più spazio nei bordi.
\item \textbf{Studia prima e poi vai a lezione}: Non è un metodo adatto a tutti i corsi, ma quando serve, \textbf{ti cambia la vita}.\\
  Se non hai già un'idea di cosa sta spiegando il professore, \textbf{rischi di perderti dopo i primi 10 minuti}. (vedi Calcolo Scientifico).\\
  In questo modo, la lezione diventa un approfondimento, non un'esposizione di cose incomprensibili.
\end{itemize}
In generale, \textbf{se hai bisogno di conoscenze pregresse per seguire la lezione}, conviene studiarti quei concetti \textbf{prima}, così quando vai a lezione:
\begin{itemize}
\item capisci davvero cosa viene spiegato,
\item riesci a seguire il filo logico,
\item puoi fare domande più precise.
\end{itemize}
Puoi usare anche un mix di questi approcci, dipende da quale ti viene più comodo e ti fa risparmiare più tempo:\\
Ogni metodo ha pro e contro. Sta a te valutarli \textbf{DOPO averli provati.} Non giudicare a priori un metodo prima di averlo testato.


\subsection{Le penne cancellabili}
Se pensi che prendere appunti per te sia importante e utile, allora tanto vale farlo bene.\\
Per questo ti consiglio di usare le penne cancellabili. Lascia perdere il pregiudizio che siano ``da bambini''.\\
Ti permettono di \textbf{scrivere, sbagliare e correggere al volo} senza dover riscrivere tutto da capo o perdere ore a ricopiare gli appunti in bella copia o aspettare che si asciughi il bianchetto.\\
\textbf{Non c'è tempo per il bianchetto.}
\begin{quote}
\textbf{Se esiste un sistema efficace per risparmiare tempo, usalo.}
\end{quote}
Una volta che le provi, ti chiedi perché non l'hai fatto prima.\\
\textbf{Provale}, male che va smetti di usarle.\\
Le migliori che ho trovato sono le \textbf{Pilot Frixion}. Costano un po' e le mine non durano tanto, ma ne vale la pena se te le puoi permettere.
\begin{quote}
\textbf{Attenzione: non lasciare i quaderni sotto il sole o vicino a fonti di calore intenso, o rischi che gli appunti svaniscano, letteralmente.}
\end{quote}


\section{ChatGPT e company}
ChatGPT e gli altri Large Language Models possono darti una grossa mano nello studio \textbf{se li usi correttamente.}\\
Purtroppo al momento non sono ancora il massimo per le task matematiche complesse, ma spero che quando leggerai questo file la situazione sia migliorata.


\subsection{Come usare l'IA per studiare}
\begin{quote}
\textbf{L'IA è uno strumento, non una scorciatoia per evitare di studiare.}
\end{quote}
Evita assolutamente di:
\begin{itemize}
\item fidarti ciecamente delle risposte senza verificarle,
\item usarla per argomenti che non hai minimamente studiato,
\item farti fare gli esercizi d'esame senza aver prima capito i concetti fondamentali
\end{itemize}


\subsection{Le allucinazioni}
\begin{quote}
\textbf{L'IA sbaglia spesso, ma è convincente.}
\end{quote}
Se non hai capito le cose che stai studiando, potrebbe tranquillamente convincerti di una cosa completamente sbagliata. Quindi:
\begin{itemize}
\item verifica \textbf{sempre} le risposte con altre fonti,
\item se qualcosa non ti torna, chiedi conferma a compagni o prof,
\item usa l'IA solo dopo aver studiato almeno le basi dell'argomento,
\item studia in gruppo. Se ha ingannato te, non è detto che riesca a ingannare anche i tuoi compagni.
\end{itemize}


\subsection{Strumenti specifici che funzionano}
\begin{itemize}
\item \textbf{GPT con WolframAlpha:} Molto più affidabile per calcoli e problemi matematici, perché utilizza l'API di WolframAlpha (anche se a volte sbaglia anche lui)
\item \textbf{Photomath:} Per controllare i passaggi degli esercizi
\item \href{https://it.symbolab.com/}{Symbolab}: Alternativa solida per step-by-step
\item \href{https://www.desmos.com/calculator?lang=it}{Desmos}: Super utile, soprattutto per i grafici delle funzioni di Analisi e Calcolo Scientifico
\end{itemize}


\subsection{Prompting efficace}
Se devi usare l'IA, almeno usala bene. Il modo in cui formuli la domanda fa una differenza enorme.
\begin{itemize}
\item \href{https://github.com/anthropics/prompt-eng-interactive-tutorial}{Repository ufficiale di Anthropic} per imparare il prompting
\item Se sei pigro: dai la repository al tuo LLM preferito e chiedigli di creare prompt per la tua task specifica.
\end{itemize}


\subsection{Come creare flashcards con l'IA}
\begin{itemize}
\item Scrivi un prompt efficace,
\item processa \textbf{un capitolo per volta} per non diluire troppo la risposta dell'IA,
\item controlla sempre che domande e risposte abbiano senso,
\item personalizza le flashcards aggiungendo esempi tuoi,
\item testa le flashcards con i compagni per vedere se sono chiare,
\end{itemize}


\section{Come organizzare gli esami da dare}
Che tu abbia un naturale talento per lo studio delle materie matematiche o meno, dovrai fare i conti con la sessione d'esame.\\
Su \href{https://uniud.esse3.cineca.it/Home.do}{Esse3} puoi vedere le date degli esami. Il problema è che \textbf{normalmente} le date degli esami non sono disponibili durante il periodo dei corsi, o comunque vengono pubblicate relativamente tardi.\\
Questo rende particolarmente difficile organizzare gli esami fin da subito.\\
In generale, potresti essere tentato di dire:\\
\textit{``Vabbè, io provo tutti gli appelli, tanto se non li passo ho un sacco di altre possibilità.''}\\
\textbf{Questa è la cosa più sbagliata che puoi fare.}\\
Strano ma vero, questo errore lo fanno \textbf{TUTTI}.\\
Scegli con cura quali esami dare in quale sessione. Non serve a nulla provare 5 esami e passarne 1, con un voto mediocre.\\
Preparane 2 o al massimo 3 a sessione, e falli \textbf{bene}.\\
Per capire quali esami scegliere, chiedi a chi li ha già fatti prima di te, o leggi il volume 2 di questa guida.


\subsection{Come scegliere quali esami dare per sessione}
Prima di buttarti a organizzare il calendario, devi capire quali esami dare e quando darli.\\
Non tutti gli esami sono uguali, e fare le scelte sbagliate può costarti mesi.


\subsubsection{Identifica i tuoi esami killer}
Alcuni esami sono oggettivamente più tosti di altri. Te ne accorgerai subito chiedendo in giro o guardando le statistiche di superamento.\\
Gli \textbf{esami killer} sono quelli che:
\begin{itemize}
\item hanno prerequisiti matematici pesanti,
\item richiedono molto tempo per essere capiti (non solo memorizzati),
\item bloccano altri corsi importanti.
\end{itemize}
\begin{quote}
\textbf{Se puoi, cerca di non mettere troppi esami killer nella stessa sessione.}
\end{quote}
\textbf{Esempio di sessioni ben bilanciate:}\\
\textbf{Gennaio:}
\begin{itemize}
\item 1 esame killer (es. Analisi)
\end{itemize}
\textbf{Febbraio:}
\begin{itemize}
\item 1 esame di media difficoltà (es. Programmazione)
\item 1 esame più gestibile (es. Inglese)
\end{itemize}
\textbf{Giugno:}
\begin{itemize}
\item 1 esame killer (es. Architettura degli Elaboratori)
\end{itemize}
\textbf{Luglio:}
\begin{itemize}
\item 1 esame di media difficoltà (es. Matematica Discreta/ Algebra Lineare)
\end{itemize}
\textbf{Settembre:}
\begin{itemize}
\item 1 esame più gestibile (es. Fisica)
\end{itemize}
(Fatti la tua organizzazione. Questa è solo di esempio)


\subsubsection{Considera le propedeuticità}
Prima di organizzare qualunque cosa, \textbf{controlla bene le propedeuticità.}  
Alcuni esami ne sbloccano altri, e se sbagli l'ordine rischi di trovarti bloccato per un intero anno accademico.\\
\textbf{Esempio:} Se non passi Programmazione e Laboratorio, non puoi dare Algoritmi e Strutture Dati.\\
\begin{quote}
\textbf{Ricorda che uno dei requisiti necessari per poter fare il tirocinio è aver superato tutti gli esami del primo anno. Perciò non lasciarli per ultimi.}
\end{quote}


\subsection{Organizzare il calendario di studio}
Crea un foglio Excel e scrivi tutti i giorni che vanno dal 1 ottobre dell'anno in corso fino al 1 ottobre dell'anno successivo. (Se non vuoi metterci 800 anni, usa le funzionalità automatiche di Excel)\\
Vai su \href{https://www.uniud.it/it/didattica/corsi/area-scientifica/scienze-matematiche-informatiche-multimediali-fisiche/laurea/informatica/studiare/orario-lezioni}{questa pagina} e segna le date di:
\begin{itemize}
\item inizio e fine delle sessioni di esame,
\item vacanze accademiche,
\item chiusura per festività patronali,
\item tutti i giorni in cui prevedi di non studiare o di studiare meno.
\end{itemize}
Poi segna i giorni in cui studiare per ogni esame, fino alla data dell'esame.\\
Se non sai in che giorno è l'esame, supponi che sia il primo giorno della sessione d'esame.\\
Questo aiuta a renderti conto che il tempo che hai realmente a disposizione non è così tanto come sembra.\\
A volte potresti metterci 2 giorni a fare un singolo esercizio, quindi tieni un bel margine di errore.


\subsubsection{Includi buffer time}
Per ogni esame, aggiungi \textbf{almeno 7 giorni extra} oltre al tempo che credi ti serva.\\
Gli imprevisti capitano sempre: esercizi più difficili del previsto, giorni in cui non riesci a studiare, argomenti che richiedono più tempo.
\begin{quote}
\textbf{È meglio credere di avere poco tempo e ritrovarsi con del tempo in più che credere di averne tanto e restare con l'acqua alla gola.}
\end{quote}%
E se non sai/vuoi usare Excel, beh, sei un informatico, sono certo che troverai una soluzione.


\section{Come prepararsi per un esame}
Usa metodi di studio efficienti, ma soprattutto, \textbf{fai esercizi d'esame} fin da subito, non aspettare la sessione di esami per esercitarti.\\
La maggior parte dei professori mette a disposizione:
\begin{itemize}
\item prove vecchie,
\item esercizi svolti.
\end{itemize}
E se non trovi materiali, o non riesci a fare un esercizio:
\begin{itemize}
\item chiedi a un compagno,
\item torna sulla teoria,
\item cerca online,
\item chiedi su Discord e nei gruppi Whatsapp,
\item vai dal docente.
\end{itemize}
\begin{quote}
\textbf{Se manca una settimana all'esame e non hai ancora provato a fare vecchi esaami, sei in ritardo.}
\end{quote}
A volte può essere utile cronometrarsi durante lo svolgimento di una vecchia prova d'esame (vedi quiz di Logica Matematica).

\clearpage
\section{Cosa fare se non passi un esame}
Fallire un esame fa parte del gioco, soprattutto qui ad Informatica. Il problema non è sbagliare, ma come reagisci all'errore.\\
La prima cosa da fare è accettare che gli esami di informatica sono spesso più difficili di quelli di altri corsi di laurea.\\
Non sei il primo e non sarai l'ultimo a fallire un esame al primo tentativo.  
Molti degli studenti più bravi che conosco hanno fallito esami importanti.


\subsection{Analizza cosa è andato storto}
Prima di riprogrammare tutto, fermati, rifletti e cerca di capire qual è il problema:
\begin{itemize}
\item Avevi davvero studiato abbastanza o ti sei illuso?
\item Hai sottovalutato la difficoltà dell'esame?
\item Hai solo letto la teoria senza fare esercizi o hai usato un metodo di studio inefficace?
\item Sapevi le cose ma ti sei bloccato durante l'esame?
\item Hai ignorato la mia guida pensando fossero consigli inutili? Male. male male.
\end{itemize}
Sii onesto con te stesso. Se non capisci il motivo del fallimento, lo ripeterai.


\subsection{Riorganizza il piano senza sovraccaricare le sessioni successive}
Dopo un fallimento, la tentazione è di recuperare tutto nella sessione successiva.\\
\textbf{È un errore madornale.}\\
Se dovevi dare 3 esami e ne hai passati solo 2, \textbf{non} cercare di farne 4 nella sessione successiva.\\
Piuttosto riprova l'esame fallito + 1 esame nuovo ben preparato
\begin{quote}
\textbf{Non avere fretta, peggiori solo la situazione.}
\end{quote}


\section{Prendersi una vacanza}
Come avrai notato, ci sono appelli anche a settembre.\\
Questo significa che ad agosto ti toccherà studiare, o meglio, se vuoi davvero passare quegli esami, è difficile evitarlo.\\
Arrivato a questo punto dell'anno, ti sarai reso conto che certi esami non si passano ``per fortuna'' o studiando all'ultimo.\\
Studiare con 40 gradi e zero motivazione non è il massimo, lo so, ma lo è ancora meno dover rifare l'esame l'anno successivo.\\
E se stai pensando: ``Eh, ma io ad agosto voglio andare in vacanza''\\
Corri pure i tuoi rischi, ma considera che potresti andare in vacanza dopo l'esame di settembre, quando:
\begin{itemize}
\item fa meno caldo,
\item c'è meno gente in giro,
\item e soprattutto: hai la mente libera.
\end{itemize}
Magari, così, te la godi davvero. Pensaci.

\clearpage
\section{Dove andare a studiare}
Se non sai dove andare a studiare, devi sapere che hai un po' di alternative tra cui scegliere.\\
Qui torna molto comodo l'applicazione EasyUniud. Usala per cercare della aule in cui non ci sono lezioni.\\
Sentiti libero di andarci. Sono lì per te.
\begin{itemize}
\item Le aule migliori interne all'università secondo me sono quelle da A034 ad A037.
\item Non disdegnare le aule da C1 a C11. Sono grandi ma le cattedre sono molto comode.
\item Una buona alternativa sono i laboratori A029 e A030.\\
	Capita spesso che alcuni professori prenotino entrambi i laboratori ma ne usino soltanto uno, quindi anche quando sull' applicazione risultano occupati entrambi, vale la pena controllare di persona.
\item Tavoli nel corridoio delle aule C.
\item Aule studio silenziose (quelle con le pareti in vetro. Ce ne sono un po' sparse per l'università).
\item I pochi tavoli ai piani superiori (sono poco illuminati e ci sono gli uffici dei professori quindi fai assoluto silenzio).
\item Biblioteca (anche se quando inizia a fare caldo non è il massimo).
\item Aule studio vicino al CUS: Sono vicino alla biblioteca, difronte la palestra del CUS. Hanno l'aria condizionata e il biliardino.
\end{itemize}
\begin{quote}
In generale cerca di non fare casino se c'è altra gente che studia.
\end{quote}


\subsection{Dove studiare quando l'università chiude}
Se anche tu, come me, ti ritrovi a dover preparare esami ad agosto, sappi che non sei solo.\\
Il problema è che in quel periodo l'università chiude presto (alle 14) e resta completamente chiusa durante la settimana di ferragosto.\\
Anche le aule studio del CUS e la biblioteca restano chiuse per tutto il mese.\\
Ma c'è una soluzione alternativa che in pochi conoscono:\\
al \textbf{Città Fiera} c'è un negozio chiamato \textbf{Giochi e Fumetti}.\\
Fuori dal negozio ci sono dei tavoli neri: \textbf{durante il weekend ci fanno i tornei}, quindi non sono utilizzabili da noi poveri studenti, ma durante la settimana restano liberi e noi ci abbiamo studiato per settimane.\\
I ragazzi del negozio sono molto disponibili e ci hanno sempre dato il permesso di usarli.\\
Certo, ci sono dei limiti:
\begin{itemize}
\item Il Città Fiera è un ambiente rumoroso (i tappi per le orecchie o le cuffie aiutano).
\item Non ci sono prese elettriche vicine ai tavoli (e qui torna utile il pc con la ricarica tramite USB-C).
\end{itemize}
Ma, d'altra parte, c'è l' aria condizionata, è facile da raggiungere, ed è uno dei pochi posti aperti in cui puoi studiare in gruppo senza essere cacciato.\\
Se sei disperato e non sai dove andare, può davvero salvarti la sessione.

\clearpage
\section{Riguardo alla copiatura}
\begin{quote}
\textbf{Copiare ti fa avanzare più velocemente, ma danneggia gli altri oltre che te stesso.}
\end{quote}
Se stai pensando di copiare, considera questo:\\
Le aziende si sono lamentate con la coordinatrice del corso del fatto che alcuni laureati in Informatica non abbiano le competenze attese.\\
E sai perché? Perché chi copia prende il titolo senza aver mai imparato davvero.\\
Questo danneggia \textbf{tutti}:
\begin{itemize}
\item Dal punto di vista dei datori di lavoro, il valore della laurea diminuisce, costringendo tutti a specializzarsi per avere qualche possibilità nel mondo del lavoro.
\item I colloqui di lavoro diventano più difficili.
\item Le aziende si fidano meno dei laureati e impostano periodi di prova più lunghi prima dell'assunzione.
\item L'università alza l'asticella, rendendo gli esami più difficili anche per chi studia onestamente perché si basa su statistiche gonfiate da chi copia.
\end{itemize}
La conseguenza è che chi copia supera anche gli esami più difficili, mentre chi studia onestamente si trova sempre più in difficoltà, le stesse difficoltà che sto cercando di mitigare creando questo documento.\\
Perciò ti avviso:
\begin{quote}
\textbf{Se sei uno che copia abitualmente, io farò di tutto per fermarti.\\
Per rispetto verso quelli che hanno dovuto trovare soluzioni per migliorarsi, non per sembrare migliori.\\
Per rispetto di quelli che si sudano ogni esame, ogni esercizio, ogni giorno.}
\end{quote}
Sappi che venire beccati a copiare può comportare l'annullamento della carriera accademica. Sicuro di voler rischiare?


\section{Conclusione}
Informatica è un percorso difficile, forse tra i più difficili, ma non è impossibile.\\
Se impari a muoverti nel modo giusto e segui i consigli di chi ci è passato prima di te, diventerà un'esperienza straordinaria che ti ripagherà.\\
Spero che questa guida ti sia d'aiuto.\\
Appena potrò rilascerò gli altri volumi. Per ora ne sono previsti altri due:
\begin{itemize}
\item \textbf{Volume 2:} Come passare gli esami (Consigli pratici e specifici per ogni esame di Informatica)
\item \textbf{Volume 3:} Come sopravvivere da studente fuori-sede
\end{itemize}

Buona fortuna!

\vspace{1cm}
\begin{flushright}
\href{https://github.com/Verryx-02}{@Verryx-02}
\end{flushright}

\subsection*{Ringraziamenti}
Un grazie speciale a chi ha letto e commentato questa guida prima della pubblicazione, aiutandomi a migliorarla, correggerla e renderla ancora più utile:
\begin{itemize}
\item Alessandro
\item \href{https://github.com/NovaActias}{NovaActias}
\item \href{https://github.com/Riccardo-Gottardi}{Riccardo-Gottardi}

\end{itemize}%
Infine, grazie alla mia band preferita, che mi ha accompagnato in questi anni. \href{https://www.youtube.com/watch?v=0dG9pXeOgT0}{Dagli un'occhiata}

\end{document}