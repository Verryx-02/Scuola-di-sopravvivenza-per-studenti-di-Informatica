\documentclass[18pt]{extarticle}
\usepackage[a4paper, top=2cm, bottom=2cm, left=2cm, right=2cm]{geometry}

\usepackage[italian]{babel}

\usepackage[
    colorlinks=true,
    linkcolor=black,        % Link interni (indice, riferimenti) in nero
    urlcolor=blue,          % Link esterni (URL) in blu
    citecolor=black,        % Citazioni in nero
    filecolor=black         % File in nero
]{hyperref} % For the Table of Contents

\usepackage{enumitem} % For lists

\usepackage{graphicx}
\usepackage{float}

\usepackage{xcolor}

\title{Scuola di sopravvivenza per Informatica volume 2}

\author{Verryx-02}
\date{2025/08/15}
\begin{document}

\maketitle

% Table of Contents (acts as your index)
\tableofcontents
\newpage

\section{Introduzione}

\subsection{Contenuto della guida}
In questa guida troverai una marea di consigli pratici per superare gli esami di Informatica. Ti consiglio di usare l'indice navigabile per andare direttamente all'esame che ti interessa.\\
Questo è il secondo documento che scrivo il \LaTeX, perciò abbi pietà di me.
Se non hai letto il Volume-1 con tutti i consigli generali per sopravvivere ad Informatica, ti consiglio caldamente di leggerlo.\\
Lo trovi nella mia \href{https://github.com/Verryx-02/Scuola-di-sopravvivenza-per-studenti-di-Informatica}{\textbf{repository Github}}. 

\subsection{Chi sono io}
Io sono \href{https://github.com/Verryx-02}{\textbf{Verryx-02}}, uno studente di Informatica fuori sede. 
Per l'esattezza, mentre sto scrivendo questo documento sono all'ultimo anno della triennale qui all'Università di Udine.\\
Da ottobre 2025 sono rappresentante degli studenti del Dipartimento di Scienze Matematiche, Informatiche e Fisiche, ma soprattutto,\\
\textbf{sono uno che ha fatto tutti gli errori possibili.}


\subsection{A chi sono rivolti questi consigli}
\textbf{A tutti.}\\
Ma in particolare a chi si sente in difficoltà, ha dubbi, o fatica a passare gli esami.\\
Se pensi che l'università sia un ostacolo insormontabile, forse la stai affrontando nel modo sbagliato e questi consigli potrebbero farti comodo.\\
Non ti mentirò. Informatica qui a Udine è dura, ma ce la puoi fare se te la giochi bene.\\
Se sei di IBML, alcuni di questi consigli sono adatti anche a te visto che hai degli esami in comune con Informatica. 


\subsection{Perché ho scritto questo documento}
Per evitare che altri commettano gli stessi errori che ho fatto io.\\  
Credo nella filosofia dell’open source: condividere ciò che imparo per me è un dovere se può rendere la vita più facile a chi verrà dopo di me.

\subsection{Perché dovresti fidarti di me}
Perché io sono partito da zero.  
Vengo dall' Istituto Tecnico Agrario di Avellino, a 800 chilometri lontano da Udine. 
A scuola non ero brillante, non avevo voglia di studiare e la mia insegnante di matematica non era un gran ché.  
Il mio livello in matematica al primo anno di Informatica si può riassumere così:
\begin{itemize}
    \item non sapevo disegnare $y = 2x$,
    \item non sapevo risolvere le equazioni,
    \item non sapevo cosa fosse una funzione,
    \item non avevo mai sentito parlare dei logaritmi,
    \item non ero in grado di superare il TOLC-S / TOLC-I, 
    \item il mio primo anno di università nonostante tutto l'impegno, ho superato 0 esami. 
\end{itemize}%
Ho dovuto adattarmi, capire quali fossero i problemi e trovare soluzioni fuori dal comune. 
Perciò puoi credermi: \textbf{Se ce l'ho fatta io, puoi farcela anche tu.}

\clearpage
\section{Disclaimer}
Questa guida è frutto dell'esperienza mia e di un ristretto gruppo di studenti di Informatica all'Università di Udine. Tutti con obiettivi, conoscenze pregresse, interessi e titoli di studio differenti.
Ti invito a non prendere il tutto come oro colato, ma allo stesso tempo, a non sottovalutare questi consigli.


\subsection{Ambito di applicazione}
Tutte le informazioni in questa guida valgono specificamente per:
\begin{itemize}
    \item \textbf{Corso di laurea in Informatica} (Università di Udine)
    \item \textbf{IBML} (per gli esami in comune)
\end{itemize}
Quando parlo di ``università'', mi riferisco esclusivamente a questi due corsi. Non posso parlare per altri corsi di laurea o altri atenei perché non li ho frequentati.


\subsection{Responsabilità}
Ho scritto questa guida di mia sponte, per condividere ciò che ritengo di aver imparato.\\
Mi assumo le piene responsabilità, delle conseguenza che questa guida può avere.\\
Non ha nessun legame con istituzioni accademiche.\\
Non rappresenta in alcun modo NEXT (partito studentesco di cui faccio parte).\\
I miei peer reviewer hanno contribuito solo con feedback e correzioni, sono pertanto esenti da ogni responsabilità.\\

\section{Contributi e aggiornamenti}
Le informazioni sono aggiornate ad \textbf{agosto 2025}. Alcuni dettagli potrebbero cambiare nel tempo.\\
Per tutti (studenti e docenti): Se questa guida ti infastidisce per qualche motivo, se noti informazioni obsolete, o semplicemente vuoi contribuire con miglioramenti, sentiti libero di:
\begin{itemize}
    \item Aprire una \href{https://github.com/Verryx-02/Scuola-di-sopravvivenza-per-studenti-di-Informatica/issues}{issue}
    \item Fare una \href{https://github.com/Verryx-02/Scuola-di-sopravvivenza-per-studenti-di-Informatica/pulls}{pull request}
\end{itemize}
Proposte di modifica e critiche sono ben accette se giustificate. Assicurati solo di seguire le indicazioni del file \href{https://github.com/Verryx-02/Scuola-di-sopravvivenza-per-studenti-di-Informatica/blob/main/CONTRIBUTING.md}{CONTRIBUTING.md}
\begin{quote}
Si assume che sappiate scrivere in \LaTeX; altrimenti, \href{https://www.learnlatex.org/it/}{sappiatelo}. \textasciitilde M.M.
\end{quote}

\clearpage
\section{Esami del primo anno}
\subsection{Architettura degli Elaboratori}
Prof. Federico Fontana

\subsection{Programmazione}
Prof. Claudio Mirolo

\subsection{Analisi Matematica}
Prof. Gianluca Gorni
Il professore di cui ho fatto l'esame non tiene più il corso. Quindi non posso aiutarti.\\ 
Posso solo dirti che una buona comprensione di Analisi ti servirà anche in altri esami.

\subsection{Matematica Discreta / Algebra Lineare}
Prof. Giuseppe Lancia

\subsection{Fisica}
Prof. Paolo Giannozzi

\noindent\rule{\textwidth}{0.6pt}

\section{Esami del secondo anno}
\subsection{Programmazione Orientata agli Oggetti}
Prof. Giorgio Brajnik

\subsection{Logica Matematica}
Prof. Alberto Marcone

\subsection{Algoritmi e Strutture Dati}
Prof.ssa Carla Piazza e Prof. Gabriele Puppis

\subsection{Sistemi Operativi}
Prof. Ivan Scagnetto e Prof.ssa Marina Lenisa

\subsection{Calcolo Scientifico}
Prof.ssa Rosanna Vermiglio

\subsection{Probabilità e Statistica}
Prof. Luigi Pace

\subsection{Fondamenti dell'Informatica}
Prof. Agostino Dovier

\noindent\rule{\textwidth}{0.6pt}

\section{Esami del terzo anno}
\subsection{Basi di Dati}
Prof. Angelo Montanari (Rettore dell'Università) e Prof. Luca Geatti

\subsection{Reti dei Calcolatori}
Prof. Marino Miculan

\subsection{Ingegneria del Software}
[Docente da specificare]

\subsection{Linguaggi di Programmazione}
[Docente da specificare]

\subsection{Interazioni Persona-Macchina}
Prof. Fabio Butussi

\section{Esami a scelta: cosa scegliere e perché}
\subsection{Internet of Things}
Prof. Ivan Scagnetto

\subsection{Machine Learning}
Prof. Giuseppe Serra

\subsection{Laboratorio di Realtà Aumentata}
Prof. Claudio Piciarelli








\section{Conclusione}
Informatica è un percorso difficile, forse tra i più difficili, ma non è impossibile.\\
Se impari a muoverti nel modo giusto e segui i consigli di chi ci è passato prima di te, diventerà un'esperienza straordinaria che ti ripagherà.\\
Spero che questa guida ti sia d'aiuto.\\
Appena potrò rilascerò anche il Volume 3: Come sopravvivere da studente fuori-sede
\\
Buona fortuna!

\vspace{1cm}
\begin{flushright}
\href{https://github.com/Verryx-02}{@Verryx-02}
\end{flushright}

\subsection*{Ringraziamenti}
Un grazie speciale a chi ha letto e commentato questa guida prima della pubblicazione, aiutandomi a migliorarla, correggerla e renderla ancora più utile:
\begin{itemize}
\item Alessandro
\item \href{https://github.com/NovaActias}{NovaActias}
\item \href{https://github.com/Riccardo-Gottardi}{Riccardo-Gottardi}

\end{itemize}%
Infine, grazie alla mia band preferita, che mi ha accompagnato in questi anni. \href{https://www.youtube.com/watch?v=0dG9pXeOgT0}{Dai un'occhiata se ti va}

\end{document}